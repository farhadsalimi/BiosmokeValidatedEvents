%% BioMed_Central_Tex_Template_v1.06
%%                                      %
%  bmc_article.tex            ver: 1.06 %
%                                       %

%%IMPORTANT: do not delete the first line of this template
%%It must be present to enable the BMC Submission system to
%%recognise this template!!

%%%%%%%%%%%%%%%%%%%%%%%%%%%%%%%%%%%%%%%%%
%%                                     %%
%%  LaTeX template for BioMed Central  %%
%%     journal article submissions     %%
%%                                     %%
%%          <8 June 2012>              %%
%%                                     %%
%%                                     %%
%%%%%%%%%%%%%%%%%%%%%%%%%%%%%%%%%%%%%%%%%


%%%%%%%%%%%%%%%%%%%%%%%%%%%%%%%%%%%%%%%%%%%%%%%%%%%%%%%%%%%%%%%%%%%%%
%%                                                                 %%
%% For instructions on how to fill out this Tex template           %%
%% document please refer to Readme.html and the instructions for   %%
%% authors page on the biomed central website                      %%
%% http://www.biomedcentral.com/info/authors/                      %%
%%                                                                 %%
%% Please do not use \input{...} to include other tex files.       %%
%% Submit your LaTeX manuscript as one .tex document.              %%
%%                                                                 %%
%% All additional figures and files should be attached             %%
%% separately and not embedded in the \TeX\ document itself.       %%
%%                                                                 %%
%% BioMed Central currently use the MikTex distribution of         %%
%% TeX for Windows) of TeX and LaTeX.  This is available from      %%
%% http://www.miktex.org                                           %%
%%                                                                 %%
%%%%%%%%%%%%%%%%%%%%%%%%%%%%%%%%%%%%%%%%%%%%%%%%%%%%%%%%%%%%%%%%%%%%%

%%% additional documentclass options:
%  [doublespacing]
%  [linenumbers]   - put the line numbers on margins

%%% loading packages, author definitions

%\documentclass[twocolumn]{bmcart}% uncomment this for twocolumn layout and comment line below
\documentclass{bmcart}

%%% Load packages
%\usepackage{amsthm,amsmath}
%\RequirePackage{natbib}
%\RequirePackage[authoryear]{natbib}% uncomment this for author-year bibliography
%\RequirePackage{hyperref}
\usepackage[utf8]{inputenc} %unicode support
%\usepackage[applemac]{inputenc} %applemac support if unicode package fails
%\usepackage[latin1]{inputenc} %UNIX support if unicode package fails


%%%%%%%%%%%%%%%%%%%%%%%%%%%%%%%%%%%%%%%%%%%%%%%%%
%%                                             %%
%%  If you wish to display your graphics for   %%
%%  your own use using includegraphic or       %%
%%  includegraphics, then comment out the      %%
%%  following two lines of code.               %%
%%  NB: These line *must* be included when     %%
%%  submitting to BMC.                         %%
%%  All figure files must be submitted as      %%
%%  separate graphics through the BMC          %%
%%  submission process, not included in the    %%
%%  submitted article.                         %%
%%                                             %%
%%%%%%%%%%%%%%%%%%%%%%%%%%%%%%%%%%%%%%%%%%%%%%%%%


\def\includegraphic{}
\def\includegraphics{}



%%% Put your definitions there:
\startlocaldefs
\endlocaldefs


%%% Begin ...
\begin{document}

%%% Start of article front matter
\begin{frontmatter}

\begin{fmbox}
\dochead{Data Note (\today)}

%%%%%%%%%%%%%%%%%%%%%%%%%%%%%%%%%%%%%%%%%%%%%%
%%                                          %%
%% Enter the title of your article here     %%
%%                                          %%
%%%%%%%%%%%%%%%%%%%%%%%%%%%%%%%%%%%%%%%%%%%%%%

\title{An Online Extensible Database of Validated Extreme Air Pollution Events for Health Research}

%%%%%%%%%%%%%%%%%%%%%%%%%%%%%%%%%%%%%%%%%%%%%%
%%                                          %%
%% Enter the authors here                   %%
%%                                          %%
%% Specify information, if available,       %%
%% in the form:                             %%
%%   <key>={<id1>,<id2>}                    %%
%%   <key>=                                 %%
%% Comment or delete the keys which are     %%
%% not used. Repeat \author command as much %%
%% as required.                             %%
%%                                          %%
%%%%%%%%%%%%%%%%%%%%%%%%%%%%%%%%%%%%%%%%%%%%%%

\author[
   addressref={aff1},                   % id's of addresses, e.g. {aff1,aff2}
   corref={aff1},                       % id of corresponding address, if any
%   noteref={n1},                        % id's of article notes, if any
   email={ivan.hanigan@anu.edu.au}   % email address
]{\inits{IC}\fnm{Ivan C.} \snm{Hanigan}}
\author[
   addressref={aff2},
   email={fay.johnston@utas.edu.au}
]{\inits{FH}\fnm{Fay H.} \snm{Johnston}}
\author[
   addressref={aff3},
   email={geoff.morgan@nsw.gov.au}
]{\inits{GG}\fnm{Geoffrey G.} \snm{Morgan}}
\author[
   addressref={aff2},
   email={grant.williamson@utas.edu.au}
]{\inits{GW}\fnm{Grant J.} \snm{Williamson}}
\author[
   addressref={aff2},
   email={Farhad.Salimi@utas.edu.au}
]{\inits{FS}\fnm{Farhad} \snm{Salimi}}
\author[
   addressref={aff4},
   email={Sarah.Henderson@bccdc.ca}
]{\inits{SH}\fnm{Sarah B.} \snm{Henderson}}

%%%%%%%%%%%%%%%%%%%%%%%%%%%%%%%%%%%%%%%%%%%%%%
%%                                          %%
%% Enter the authors' addresses here        %%
%%                                          %%
%% Repeat \address commands as much as      %%
%% required.                                %%
%%                                          %%
%%%%%%%%%%%%%%%%%%%%%%%%%%%%%%%%%%%%%%%%%%%%%%

\address[id=aff1]{%                           % unique id
  \orgname{National Centre for Epidemiology and Population Health, Australian National University}, % university, etc
  \street{Eggleston Road},                     %
  %\postcode{}                                % post or zip code
  \city{Canberra},                              % city
  \cny{AU}                                    % country
}
\address[id=aff2]{%
  \orgname{Menzies School of Population Health, University of Tasmania},
  \street{},
  \postcode{}
  \city{Hobart},
  \cny{AU}
}
\address[id=aff3]{%
  \orgname{University Centre for Rural Health, University of Sydney},
  \street{},
  \postcode{}
  \city{Sydney},
  \cny{AU}
}
\address[id=aff4]{%
  \orgname{School of Population and Public Health, University of British Columbia},
  \street{},
  \postcode{}
  \city{Vancouver},
  \cny{CA}
}


%%%%%%%%%%%%%%%%%%%%%%%%%%%%%%%%%%%%%%%%%%%%%%
%%                                          %%
%% Enter short notes here                   %%
%%                                          %%
%% Short notes will be after addresses      %%
%% on first page.                           %%
%%                                          %%
%%%%%%%%%%%%%%%%%%%%%%%%%%%%%%%%%%%%%%%%%%%%%%

%\begin{artnotes}
%\note{Sample of title note}     % note to the article
%\note[id=n1]{Equal contributor} % note, connected to author
%\end{artnotes}

\end{fmbox}% comment this for two column layout
%%%%%%%%%%%%%%%%%%%%%%%%%%%%%%%%%%%%%%%%%%%%%%
%%                                          %%
%% The Abstract begins here                 %%
%%                                          %%
%% Please refer to the Instructions for     %%
%% authors on http://www.biomedcentral.com  %%
%% and include the section headings         %%
%% accordingly for your article type.       %%
%%                                          %%
%%%%%%%%%%%%%%%%%%%%%%%%%%%%%%%%%%%%%%%%%%%%%%

\begin{abstractbox}

\begin{abstract} % abstract
\parttitle{Background} %if any
Epidemiological studies of the health effects of
biomass smoke events (such as bushfires or wood-heater smoke spikes due
to inversion layers) have been hampered by the lack of availability of
datasets that explicitly pertain to these sources. Extreme air pollution
events may also be caused by dust storms, fossil fuel induced smog
events or factory fires. This paper presents an open and extensible
database developed by the authors to identify historical spikes in PM
concentrations and to evaluate whether they were caused by vegetation
fire smoke or by other possible sources. These methods provide a
systematic framework for retrospective identification of the air quality
impacts of biomass smoke in a region that is seasonally affected by
fires. In this paper, we describe the database and data aquisition
methods, as well as analytical considerations when validating historical
events using a range of reference types.

\parttitle{Methods} %if any
Several major urban centers and smaller regional towns
in the Australian states of New South Wales, Western Australia, and
Tasmania were selected as they are intermittently affected by extreme
episodes of vegetation fire smoke. Air pollution data was collated and
missing values were imputed. Extreme values were identified and a range
of sources of reference information were assessed for each date.
Reference types online newspaper archives, government and research
agency records, satellite imagery and a Dust Storms database.

\parttitle{Results}
This dataset contains validated events of extreme
biomass smoke pollution across Australian cities. The authors have
previously demonstrated the utility of this database in analyses of
hospital admissions and mortality data for these locations to quantify
the pollution-related health effects of these events.

\parttitle{Conclusions}
The database was created using open source
software and this makes the prospect for future extensions to the
database possible. This is because if other scientists notice an
ommision or error in these data they can offer an amendment. We believe
that this will improve the database and benefit the whole biomass smoke
health research community.

\end{abstract}

%%%%%%%%%%%%%%%%%%%%%%%%%%%%%%%%%%%%%%%%%%%%%%
%%                                          %%
%% The keywords begin here                  %%
%%                                          %%
%% Put each keyword in separate \kwd{}.     %%
%%                                          %%
%%%%%%%%%%%%%%%%%%%%%%%%%%%%%%%%%%%%%%%%%%%%%%

\begin{keyword}
\kwd{sample}
\kwd{article}
\kwd{author}
\end{keyword}

% MSC classifications codes, if any
%\begin{keyword}[class=AMS]
%\kwd[Primary ]{}
%\kwd{}
%\kwd[; secondary ]{}
%\end{keyword}

\end{abstractbox}
%
%\end{fmbox}% uncomment this for twcolumn layout

\end{frontmatter}

%%%%%%%%%%%%%%%%%%%%%%%%%%%%%%%%%%%%%%%%%%%%%%
%%                                          %%
%% The Main Body begins here                %%
%%                                          %%
%% Please refer to the instructions for     %%
%% authors on:                              %%
%% http://www.biomedcentral.com/info/authors%%
%% and include the section headings         %%
%% accordingly for your article type.       %%
%%                                          %%
%% See the Results and Discussion section   %%
%% for details on how to create sub-sections%%
%%                                          %%
%% use \cite{...} to cite references        %%
%%  \cite{koon} and                         %%
%%  \cite{oreg,khar,zvai,xjon,schn,pond}    %%
%%  \nocite{smith,marg,hunn,advi,koha,mouse}%%
%%                                          %%
%%%%%%%%%%%%%%%%%%%%%%%%%%%%%%%%%%%%%%%%%%%%%%

%%%%%%%%%%%%%%%%%%%%%%%%% start of article main body
% <put your article body there>

%%%%%%%%%%%%%%%%
%% Background %%
%%

%\section*{Background}

%Following the other paper.

\section*{Findings}




\section{Abstract}\label{abstract}

\textbf{Background:} Epidemiological studies of the health effects of
biomass smoke events (such as bushfires or wood-heater smoke spikes due
to inversion layers) have been hampered by the availability of datasets
that explicitly pertain to these sources. Extreme air pollution events
may also be caused by dust storms, fossil fuel induced smog events or
factory fires. This paper presents an open and extensible database
developed by the authors to identify historical spikes in PM
concentrations and to evaluate whether they were caused by vegetation
fire smoke or by other possible sources. These methods provide a
systematic framework for retrospective identification of the air quality
impacts of biomass smoke in a region that is seasonally affected by
fires. In this paper, we describe the database and data aquisition
methods, as well as analytical considerations when validating historical
events using a range of reference types.

\textbf{Methods:} Several major urban centers and smaller regional towns
in the Australian states of New South Wales, Western Australia, and
Tasmania were selected as they are intermittently affected by extreme
episodes of vegetation fire smoke. Air pollution data was collated and
missing values were imputed. Extreme values were identified and a range
of sources of reference information were assessed for each date.
Reference types online newspaper archives, government and research
agency records, satellite imagery and a Dust Storms database.

\textbf{Results:} This dataset contains validated events of extreme
biomass smoke pollution across Australian cities. The authors have
previously demonstrated the utility of this database in analyses of
hospital admissions and mortality data for these locations to quantify
the pollution-related health effects of these events.

\textbf{Conclusions:} The database was created using open source
software and this makes the prospect for future extensions to the
database possible. THis is because if other scientists notice an
ommision or error in these data they can offer an amendment. We believe
that this will improve the database and benefit the whole biomass smoke
health research community.

\section{Findings}\label{findings}

\subsection{Description}\label{description}

The background and purpose of the database or data collection should be
presented for readers without specialist knowledge in that area. For
this database we should cite the original paper by Johnston et al.
(2011a) as well as the two health analyses of Hospitalisation (Martin
\emph{et al.} 2013) and Mortality (Johnston \emph{et al.} 2011b).

This will be followed by a brief description of the protocol for data
collection, data curation and quality control, and what is being
reported in the article.

The user interface should be described and a discussion of the intended
uses of the database, and the benefits that are envisioned, should be
included, together with data on how it compares with similar existing
databases. A case study of the use of the database may be presented. The
planned future development of new features, if any, should be mentioned.

The findings section can be broken into subsections with short
informative headings. There is no maximum length for this section but we
encourage authors to be concise.

\subsection{General Protocols}\label{general-protocols}

For each location, up to 13 yr (between 1994 and 2007) of daily air
quality data measured asPMless than 10um (PM10 ) or less than 2.5 um
(PM2.5 ) in aerodynamic diameter were examined. Air pollution data were
pro- vided by government agencies in the states of Western Australia,
New South Wales, and Tasmania. Daily averages for each site were
calculated excluding days with less than 75\% of hourly measurements. In
Sydney and Perth, where data were collected from several monitoring sta-
tions, the missing daily site-specific PM10 and PM2.5 con- centrations
were imputed using available data from other proximate monitoring sites
in the network. The daily city-wide PM10 and PM2.5 concentrations were
then estimated following the protocol of the Air Pollution and Health: a
European Approach studies (Atkinson \emph{et al.} 2001).

\subsection{Detailed Data Collation and Validation
Methods}\label{detailed-data-collation-and-validation-methods}

First a `filling-in' procedure was used to improve data completeness. It
entailed the substitution of the missing daily values with a weighted
average of the values from the rest of the monitoring stations. The
pollutant measures from all stations providing data were then averaged
to provide single, city-wide estimates of the daily levels of the
pollutants

For each city, all days in which PM10 or PM2.5 exceeded the 95th
percentile were identified over the entire time series. These extreme
values were termed `events'. A range of sources was ex- amined to
identify the cause of particulate air pollution events, including
electronic news archives, Internet searches for other reports,
government and research agencies, satellite imagery and a Dust Storms
database. Also examined were remotely sensed aerosol optical thickness
(AOT) data to provide further information about days for which the other
methods

Step 1. Source air pollution data. Both time series observations and
spatial data regarding site locations.

Step 1.1. NSW data downloaded from an online data server. Site locations
(Lat and Long) obtained from website.

Step 1.2. WA data sent on CD from contacts at the WA Government
Department, these were hourly data as provided. Cleaned so as only days
with \textgreater{}75\% of hours are used.

Step 1.3. Tasmanian data sent via email from contact at the Department,
these were daily data.

Step 1.4. All data combined and Quality Control checked in the PostGIS
database.

Step 2. Spatial data for cities.

Step 3. Calculate a network average. In cities where data were collected
from several monitoring sta- tions, the missing daily site-specific PM
concentrations were imputed using available data from other proximate
monitoring sites in the network. The daily city-wide PM concentrations
were then estimated following the protocol of the Air Pollution and
Health: a European Approach studies. Atkinson et al. (2001).

Step 3.1. Prepare Data. First it was necessary to find the minimum date
that the series of continuous observations can be considered to start.
In the Australian datasets the initial observations could not be used
because the were sometimes only one day per week, only during a
particular season or of poor quality due to teething problems with
equipment and procedures. Then it was necessary to identify missing
dates. Get a list of the sites to include -- that is with more than 70\%
observed over the time period (as defined after assessing min and max
dates of period).

Step 3.2. Loop over each station individually and calculate a daily
network average of all the other non-missing sites (ie an average of all
stations except the focal station of that iteration in the loop).

Step 3.3. Calculate three monthly seasonal mean of these non-missing
stations. Calculate a three-month seasonal mean for MISSING site.
Estimate missing days at missing sites.

Step 3.4. Join all sites for city wide averages and fill any missing
days with avg of before and after.

Step 3.5 Take the average of all sites per day for city wide averages.

Step 3.6. Fill any missing days with avg of before and after (if this is
less than 5\% of days).

Step 4. Validate events and identify the causes. Select any events with
PM10 or PM2.5 greater than 95 percentile. Manually validate events using
online newspaper archives, government and research agency records,
satellite imagery and other sources (such as a Dust Storm database).
Enter the information for each event into the custom built data entry
forms. For any events with references for multiple types of source,
assess the liklihood of any single source being the dominant source.
Double check any remaining 99th percentile dates with no references.

\subsection{Availability and
requirements}\label{availability-and-requirements}

Lists the following:

\begin{itemize}
\itemsep1pt\parskip0pt\parsep0pt
\item
  Project name: BiosmokeValidatedEvents
\item
  \textbf{http://swish-climate-impact-assessment.github.io/BiosmokeValidatedEvents/}
\item
  Operating system(s): R package is platform independent. Data Entry
  forms are Microsoft Windows.
\item
  Programming language: R and SQL
\item
  Other requirements: PostgreSQL (PostGIS is desirable)
\item
  License: CC BY 4.0
\item
  Any restrictions to use: amendments of errors of ommision or
  commission are invited but will be vetted before insertion into the
  master database.
\end{itemize}

\subsection{Availability of supporting
data}\label{availability-of-supporting-data}

BMC Research Notes encourages authors to deposit the data set(s)
supporting the results reported in submitted manuscripts in a
publicly-accessible data repository, when it is not possible to publish
them as additional files. This section should only be included when
supporting data are available and must include the name of the
repository and the permanent identifier or accession number and
persistent hyperlink(s) for the data set(s). The following format is
required:

``The data set(s) supporting the results of this article is(are)
available in the {[}repository name{]} repository, {[}unique persistent
identifier and hyperlink to dataset(s).''

Where all supporting data are included in the article or additional
files the following format is required:

``The data set(s) supporting the results of this article is(are)
included within the article (and its additional file(s))''

We also recommend that the data set(s) be cited, where appropriate in
the manuscript, and included in the reference list.

A list of available scientific research data repositories can be found
here. A list of all BioMed Central journals that require or encourage
this section to be included in research articles can be found here.


\section{References}

Atkinson, R.W., Anderson, R.H., Sunyer, J., Ayres, J., Baccini, M.,
Vonk, J.M., Boumghar, A., Forastiere, F., Forsberg, B., Touloumi, G.,
Schwartz, J. \& Katsouyanni, K. (2001). Acute Effects of Particulate Air
Pollution on Respiratory Admissions. \emph{American Journal of
Respiratory and Critical Care Medicine}, \textbf{164}, 1860--1866.

Johnston, F., Hanigan, I., Henderson, S., Morgan, G. \& Bowman, D.
(2011a). Extreme air pollution events from bushfires and dust storms and
their association with mortality in Sydney, Australia 1994-2007.
\emph{Environmental Research}, \textbf{111}, 811--816.

Johnston, F.H., Hanigan, I.C., Henderson, S.B., Morgan, G.G., Portner,
T., Williamson, G.J. \& Bowman, D.M.J.S. (2011b). Creating an integrated
historical record of extreme particulate air pollution events in
Australian cities from 1994 to 2007. \emph{Journal of the Air \& Waste
Management Association (1995)}, \textbf{61}, 390--398.

Martin, K.L., Hanigan, I.C., Morgan, G.G., Henderson, S.B. \& Johnston,
F.H. (2013). Air pollution from bushfires and their association with
hospital admissions in Sydney, Newcastle and Wollongong, Australia
1994-2007. \emph{Australian and New Zealand Journal of Public Health},
\textbf{37}, 238--243.



The LaTeX template needs bibtex style citations, so here is one to ensure the compiler works while creating drafts.  The main paper to cite is \cite{Johnston2011c}.

\section*{Instructions for Accessing the Database}

The Database can be accessed by the pgAdmin tool for PostgreSQL databases, the R software or by using ODBC and MS Access.  The latter method is the recommended way to view the data entries using Forms stored in the MS Access database provided with the downloadable materials.  A Password is available on request to the corresponding author.

An additional document shows the instructions to access the database in more detail [see Additional file 1].
%%%%%%%%%%%%%%%%%%%%%%%%%%%%%%%%%%%%%%%%%%%%%%
%%                                          %%
%% Backmatter begins here                   %%
%%                                          %%
%%%%%%%%%%%%%%%%%%%%%%%%%%%%%%%%%%%%%%%%%%%%%%

\begin{backmatter}

\section*{Competing interests}
  The authors declare that they have no competing interests.

\section*{Author's contributions}
    Text for this section \ldots

\section*{Acknowledgements}
  Text for this section \ldots
%%%%%%%%%%%%%%%%%%%%%%%%%%%%%%%%%%%%%%%%%%%%%%%%%%%%%%%%%%%%%
%%                  The Bibliography                       %%
%%                                                         %%
%%  Bmc_mathpys.bst  will be used to                       %%
%%  create a .BBL file for submission.                     %%
%%  After submission of the .TEX file,                     %%
%%  you will be prompted to submit your .BBL file.         %%
%%                                                         %%
%%                                                         %%
%%  Note that the displayed Bibliography will not          %%
%%  necessarily be rendered by Latex exactly as specified  %%
%%  in the online Instructions for Authors.                %%
%%                                                         %%
%%%%%%%%%%%%%%%%%%%%%%%%%%%%%%%%%%%%%%%%%%%%%%%%%%%%%%%%%%%%%

% if your bibliography is in bibtex format, use those commands:
\bibliographystyle{bmc-mathphys} % Style BST file (bmc-mathphys, vancouver, spbasic).
\bibliography{bmc_article}      % Bibliography file (usually '*.bib' )
% for author-year bibliography (bmc-mathphys or spbasic)
% a) write to bib file (bmc-mathphys only)
% @settings{label, options="nameyear"}
% b) uncomment next line
%\nocite{label}

% or include bibliography directly:
% \begin{thebibliography}
% \bibitem{b1}
% \end{thebibliography}

%%%%%%%%%%%%%%%%%%%%%%%%%%%%%%%%%%%
%%                               %%
%% Figures                       %%
%%                               %%
%% NB: this is for captions and  %%
%% Titles. All graphics must be  %%
%% submitted separately and NOT  %%
%% included in the Tex document  %%
%%                               %%
%%%%%%%%%%%%%%%%%%%%%%%%%%%%%%%%%%%

%%
%% Do not use \listoffigures as most will included as separate files

\section*{Figures}
  \begin{figure}[h!]
  \caption{\csentence{Sample figure title.}
      A short description of the figure content
      should go here.}
      \end{figure}

\begin{figure}[h!]
  \caption{\csentence{Sample figure title.}
      Figure legend text.}
      \end{figure}

%%%%%%%%%%%%%%%%%%%%%%%%%%%%%%%%%%%
%%                               %%
%% Tables                        %%
%%                               %%
%%%%%%%%%%%%%%%%%%%%%%%%%%%%%%%%%%%

%% Use of \listoftables is discouraged.
%%
\section*{Tables}
\begin{table}[h!]
\caption{Sample table title. This is where the description of the table should go.}
      \begin{tabular}{cccc}
        \hline
           & B1  &B2   & B3\\ \hline
        A1 & 0.1 & 0.2 & 0.3\\
        A2 & ... & ..  & .\\
        A3 & ..  & .   & .\\ \hline
      \end{tabular}
\end{table}

%%%%%%%%%%%%%%%%%%%%%%%%%%%%%%%%%%%
%%                               %%
%% Additional Files              %%
%%                               %%
%%%%%%%%%%%%%%%%%%%%%%%%%%%%%%%%%%%

\section*{Additional Files}
  \subsection*{Additional file 1 --- Sample additional file title}
    Additional file descriptions text (including details of how to
    view the file, if it is in a non-standard format or the file extension).  This might
    refer to a multi-page table or a figure.

  \subsection*{Additional file 2 --- Sample additional file title}
    Additional file descriptions text.


\end{backmatter}
\end{document}
